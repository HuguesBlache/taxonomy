\section{Introduction}
%This is the introduction
%Hakim
\acrfull{ITS} are one of the cornerstones of smart cities. Beyond these buzzwords are technological solutions to better answer the needs of all citizens, to move in a timely and efficient manner while minimizing their negative impacts such as road crashes and air pollution. 
% It is in fact inconceivable to imagine a smart city without an ITS.
Several of these solutions are driven by advancements in telecommunication systems. The most recent depend on the ongoing deployment of 5G technology enabling enhanced mobile broadband, ultra-reliable, low latency and massive machine communications. New telecommunication technologies open the doors for a wide variety of \acrfull{ITS} applications to be implemented, in particular for users, vehicles and the road infrastructure to become connected. 
These \acrfull{CT} applications, many of which have been proposed for some time, will share the same transportation and telecommunication infrastructure. Therefore, they will be competing for transportation and telecommunication resources. Moreover, from the telecommunication side, the competition will also come from other verticals (i.e., power, safety, finance, health, etc.) sharing the same infrastructure.

There is also competition among telecommunication technologies for transportation applications, in particular for vehicular communications. Several countries allocated spectra around 5.9~GHz, to be used by \acrfull{DSRC} in the US and ETSI-G5 in the European Union, both based on IEEE 802.11p but incompatible. 20~years after the allocation to \acrshort{DSRC} in the US in 1999, few vehicles or \acrfull{RSU}s support these technologies, which in turn does not provide any incentive for manufacturers to support the technology and users to buy devices equiped with the technology. The spectrum allocation is thus under threat by other industries for other uses, which has culminated in the US with the decision in November 2020 of the Federal Communications Commission (FCC) to reallocate the lower 45 MHz half of the DSRC spectrum for Wi-Fi and other unlicensed uses since the car industry failed to make use of the technology for its intended safety purposes. The FCC supports to use the remaining 30~MHz currently allocated to DSRC for cellular based vehicular communication based on 4G and 5G~\cite{brodkin_fcc_2020}. %{\bf ajouter ref https://arstechnica.com/tech-policy/2020/11/fcc-adds-45mhz-to-wi-fi-promising-supersize-networks-on-5ghz-band/}
% FCC declared that their decision ``begins the transition away from DSRC services—which are incompatible with C-V2X—to hasten the actual deployment of ITS [Intelligent Transportation Systems] services that will improve automotive safety.
In July 2019, the EU announced the adoption of a technology-neutral approach to vehicular communications, leaving the door open to cellular networks. 

This is why a thorough understanding of the transportation application characteristics alongside telecommunication issues is needed in order to guarantee the smooth operation of the applications. 
Several questions can be asked to study the performance of different types of connected transportation applications. The first is how to characterize the Key Performance Indicators (\acrshort{KPI}s) for each application to work properly. Some applications may be sensitive to latency, whereas others may be affected by information loss. The second question relates to the acceptable levels of these \acrshort{KPI}s, i.e.\ the application \acrshort{KPI} requirements. Regarding latency, for instance, some applications need seconds or minutes of response time whereas others require milliseconds. Finally, once the applications, \acrshort{KPI}s and \acrshort{KPI} requirements are well defined, one may wonder what type of telecommunication infrastructure would be most appropriate for various application deployments. 

%We believe that in this complex context that maps intelligent transportation in the telecommunication infrastructure,
A comprehensive and thorough study and classification of the \acrshort{KPI} requirements of \acrshort{CT} applications must therefore be carried out. The objective is to quickly identify the needs, constraints and potential telecommunication technology that can be associated to different categories of applications. To that end, a comprehensive taxonomy is an essential tool. While the literature does contain a few papers with classifications and taxonomies, they typically focus only on one side of the coin, either on the transportation or the telecommunication side. Yet, as a new transportation system paradigm is dawning on us, a more holistic understanding of both systems is required to establish the requirements as well as the priorities of the different \acrshort{CT} applications. This system comprehension is further driven by the migration of \acrshort{CT} applications to smartphones and the use of cellular networks.

The objective of this paper is therefore to map \acrshort{CT} applications on telecommunication technologies and to provide a comprehensive taxonomy of \acrshort{CT} applications based on their characteristics relevant for telecommunication technologies. The paper is organized as follows: the next section presents previous research on \acrshort{CT} application taxonomies. Section~\ref{sec:methodology} covers the methodology used to search the literature, identify the relevant documents and build the taxonomy. Section~\ref{sec:taxonomy} presents the resulting taxonomy and section~\ref{sec:conclusion} concludes this work. 
