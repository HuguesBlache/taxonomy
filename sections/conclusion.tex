\section{Conclusion}\label{sec:conclusion}
The large review of the transportation and telecommunication literature undertaken in this work allowed us to build a taxonomy of \acrshort{CT} applications and to map existing telecommunication technologies to these applications. The resulting data is rich and is made available to other researchers to validate and build upon (see GitHub repository~\url{https://github.com/HuguesBlache/taxonomy}).

The results show that cellular networks, in particular 5G and even LTE, can meet the needs of all applications, as far as one can find in the literature. This ``universal'' suitability of cellular networks draws a sharp contrast with communication technologies developed specifically for transportation applications, chiefly \acrshort{DSRC}. This can be seen as the logical conclusion of historical trends: the rapid evolution and improvements of consumer technologies, the convergence of various devices as applications on the same device, in particular smartphones, and the consumerization of information technologies, i.e.\ the proliferation of devices originating in the consumer market to be used for professional and industrial purposes. 

As noted before, several \acrshort{CT} applications included in the taxonomy like instant messaging, routing and maps, internet services and media downloading are all available and mostly used on smartphones that the majority of users carry everywhere with them, while in-vehicle computers provide limited and less appealing interfaces for some of these applications, and often simply duplicate the same services by connecting to smartphones. Even more critical applications may be moving to smartphones as recent studies consider using smartphones as on-board units~\cite{abdel-aty_using_2020}. 
%This seems to mirror the rise of interest in cellular networks, and the conclusion of this 

This migration is mirrored by the history of the spectrum allocation for vehicular applications summed up in the introduction. The logical conclusion is that cellular networks, currently LTE and soon 5G, support most if not all \acrshort{CT} applications, while other technologies, notably DSRC, do not support as many applications and would require large investments to compete. 

While the \acrshort{CT} applications were generally presented in this paper from a user-centric perspective, the literature review has clearly shown that most \acrshort{CT} applications still focus on motorized vehicles. Yet, considering current transportation trends like the emergence of ``mobility as a service (MaaS)'' and access to vehicles and services replacing ownership, the transportation system is expected to become more user-centric to meet the mobility needs of each user while minimizing the negative impacts of transportation. This will reinforce the shift to smartphones as the platform of choice for \acrshort{CT} applications, as users carry their smartphones across various modes of transportation. This leaves the question of the \acrshort{CT} applications that will remain on vehicles. The core of such applications will be safety and efficiency applications that automatically take some form of control of vehicles, or may do so in the future to increase their performance. While all informative applications may move to smartphones, their integration as smartphone applications in the vehicle requires some thought on user interface and ergonomy so that they will be taken into account as vehicle dashboards become more digital and display more information. Others may remain on the vehicle as they benefit from access to the vehicle data and to other ways to provide information to the user, e.g.\ through haptic feedback. Finally, as vehicles become more automated, they require more data about their environment, which may bring back other informative applications on the vehicle. Improved inter-operability of smartphones and vehicles may also blur the boundary between the two, with applications running seamlessly on one or the other, and in the cloud. 

{\bf NS: ajouter des elements de perspective (future work)?}

%- We have noticed in our research that most transport applications are vehicle-centric, not user-centric. One of the points for improvement in future research will surely be to take into account the multimodal aspect of this problem and not only in terms of vehicles. This paradigm will obviously pose other problems but could potentially lead to the emergence of a new type of specific application not perceived or taken into account until now.

%% Hugues

% - Usefulness of this classification for a quick decision.

% - Talk about the historical side, that is to say that the evolution of applications is changing. Applications that have been found in the literature for decades, the objectives are not the same. For example, insurance requests no longer go through connected cars but no longer through phones (to be verified)

% - Pragmatic views on the evolution of the technologies used. That is to say that 5G is treated today but before the DRSC was more emphasized. But according to the articles, some researchers will have different paradigms, that is to say, will propose one technology rather than another.

%% NS US Department of Transportation~\cite{usdt_smartphone_2020}: utiliser reference pour argumenter que les applications migrent vers d'autres plateformes, en particulier les cellulaires
% Discuter la co-evolution rapide des technologies, des applications et des usages
% Top-down et bottom-up

%\subsection{Towards an abandonment of the DSRC?}

%Since the recent announcement by the FCC in the USA of the opening of the 5.9 GHz spectrum for unlicensed uses (Wi-Fi)\cite{fcc_fcc_2020}, initially planned for the DSRC for V2X applications, the challenges of CT applications are perceived as a zero-reset problem. Notably after twenty years of research in this field with the DSRC study.
%Several reasons are at the origin of this choice, apart from the costs of financing these projects and the political domain, most of the cases of applications have been confined to only experimental studies and a certain slowness persists among the automobile manufacturers to the introduction of DSRCs in new car models.
%By this modification of spectrum and at the same time a possible change of paradigms in the research community, the cellular network would be a possible solution to overcome this problem. This application is called this time C-V2X for Cellular Vehicles-To-Everything, will have a spectrum reserved for automotive safety applications between 5.895 and 5.925 GHz, or 30 MHz.\cite{fcc_fcc_2020}The United States is not the only example of this change, the European Union has proposed a bandwidth of 10 MHz, between 5.875 and 5.925 GHZ for wireless technologies, including 5G, concerning road safety using from the ETSI institute \cite{anonymous_harmonisation_2020}. Likewise for China, using 5G communication, which offers a bandwidth of 20 MHz, between 5905-5925 MHz\cite{qualcomm_c-v2x_2021}In this new opportunity, some companies, like Ericsson\cite{ericsson_c-v2x_2021}, are working and offering solutions for applications using 4G and 5G. These companies rely on the cellular network, via IoT, to increase safety on the roads.Despite these changes in technology and standard at the same time, the DSRC was criticized for its inactivity on the road, but the concern would it not also be the real lack of communicating car on the road? As Bezzina said, technological change will alter the architectural perspective of communication\cite{ong_pilot_2020} and possibly delay the implementation of CT applications. And those at the same time, the lack of a part of smart mobility for smart cities?However, despite the difficulty of finding equivalent standards allowing cellular and DSRC technologies to operate simultaneously, there could be a hybrid architecture that favors certain applications over certain technologies.\cite{kiela_review_2020}
%\subsection{The leak of applications on vehicles, the fault of the Digital Dashboard? }

%The debate between the use of DSRC and the cellular network, shows a point. Most network players have as an argument the implementation of safety applications also effisiency applications. So where are the other applications, at least the categories mentioned throughout our taxonomy?One of the possible legitimate questions in the case of automated, or even autonomous, cars is the question of the design of the Digital Dashboard for applications \cite{gowda_dashboard_2014}, in particular that of the infotainment application\cite{heikkinen_mobile_2013}.%Aside from the issues and the affectation of user driving behavior based on designs\cite{firby_its_2020}, the evolution of digital dashboards seems outdated compared to user perceptions. And they prefer to use their smartphones, which are more present on a daily basis for use.Indeed, in addition to the aspect of form, there is the question of applications as an interaction with machines and humans. Knowing that on average a smartphone user uses only 9 applications per day\cite{web_55_2019} and especially since among the smartphone applications, on average 25\% of them are used only once\cite{statista_research_department_mobile_2021-1}. What will happen to applications via digital dashboards?It is also possible to see a clear difference in use between the different technologies available, which the mobile phone will potentially take priority. This is for example the case of the use of the social network Facebook in comparison between the use by the mobile phone and the desktop computers\cite{tankovska_facebook_2021}. We can easily imagine that it could have the same depth of use with dashboards if it integrated instant messaging type applications for infoteinment.It is therefore possible to envisage two non-exhaustive solutions to this problem. Either the car manufacturers have improved their interfaces with the driver, even if it means competing with the manufacturers of smartphones. Or companies focus on critical automotive applications, such as safety applicationsMoreover, the perception of a replacement of the digital dashboard by smartphones can be considered in view of the evolution of the two designs and the percentages of use of the two technologies.\cite{abdel-aty_using_2020}

%\subsection{For more increased uses of smartphone applications?}

%The previous section gives a brief overview of the evolution and influence of dashboards in certain types of applications. It should be noted that a transfer of applications to smartphones is possible as mentioned in the literature review section.But another problem arises, over 50\% of smartphone users have downloaded apps\cite{statista_research_department_monthly_2021} and a large majority of the apps used are infotainment\cite{statista_research_department_mobile_2021}. A possible strategy for applications that are efficient or have a real impact on traffic will surely have to focus on a single application or even coexist with other infotainment applications, for example geolocation.However, it is very likely that some applications are not usable by smartphones, such as security applications that require increased prioritization of communications, as we have seen in this taxonomy.

{\bf NS: reste à finaliser les définitions des applications (je le ferai)} 
