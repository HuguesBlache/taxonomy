\section{Literature review}
This literature review focuses on existing taxonomies of \acrshort{CT} applications, including connected and automated vehicle applications, and how they relate to telecommunication techonologies. 
There are few taxonomies and classifications of \acrshort{ITS} or \acrshort{CT} applications in the literature~\cite{chang_estimated_2015}. For example, a taxonomy of 42 smart mobility services in nine smart cities worldwide is presented in~\cite{cledou_taxonomy_2018}. The taxonomy comprises eight dimensions: type of services, maturity level, users, applied technologies, delivery channels, benefits, beneficiaries, and common functionality. 
However, this taxonomy lacks the transportation application categories, the closest being the ``Type of service'' dimension, and does not consider important categories like safety and security for example. In addition, this taxonomy does not cover the required \acrshort{KPI}s per application. 

Another example of classification of \acrshort{ITS} vehicular communication applications is proposed by the \acrfull{ETSI} in its technical reports \cite{etsi_etsi_tr_102_638_intelligent_2009,etsi_tr_102_863_intelligent_2011}, in which \acrshort{ITS} applications and use cases are classified in three main categories: co-operative road safety, traffic efficiency and others. Other reports use more categories for the applications and consider some \acrshort{KPI}s for each application~\cite{hamida_security_2015,papadimitratos_vehicular_2009,3gpp_tr_22886_3rd_2018,al-sultan_comprehensive_2014}. Yet, the possible telecommunication technologies that can be used for each application are not presented.

There have been a few partial efforts in the literature to build a taxonomy of \acrshort{CT} applications. For example, the Internet of vehicles with its architecture is presented in~\cite{kaiwartya_internet_2016} and compared to vehicular ad hoc networks (VANETs). Applications are classified in broad categories, and the possible communication technologies are described, but only at the level of the categories and not each specific application. Also missing are the \acrshort{KPI} requirements for each application. Communication technologies are presented with a very high level of details for autonomous driving applications in~\cite{wang_networking_2019}. The \acrshort{KPI}s of the communication technologies are presented, yet without the application requirements. 
Previous work often focuses on a limited set of \acrshort{CT} applications~\cite{lee_latency_2017}, or fail to mention the usable communication technologies for all the applications~\cite{raza_social_2018,karagiannis_vehicular_2011}. Only four categories of V2X use cases and their requirements are presented at a high level in~\cite{machardy_v2x_2018}. 

%In terms of transportation, these reports aim to classify the applications in terms of transport with classes such as safety, efficiency, etc.. (even if it lacks relevant categories like comfort ...) and also to assign KPI to each application \cite{hamida_security_2015,papadimitratos_vehicular_2009,3gpp_tr_22886_3rd_2018,al-sultan_comprehensive_2014}. However, there remains a lack of technologies to deploy for each application but also of application relevance. Indeed, from year to year, the paradigms in the studies of connected vehicles change, as do the applications. We can therefore have applications that today seem obsolete, such as the Insurance and Financial Services application, which is no longer supported by the car today.

%Other articles have already started the work that we are trying to do in our research without having this primary focus in the article.
%This is the case of the article in “Internet of vehicles: motivation, layered architecture, network model, challenges and future aspects” \cite{kaiwartya_internet_2016} which classifies specific applications in terms of broad categories of transport applications that are useful to us ( security, infotainment, etc.) while studying the technologies that can be used for proper implementation. However, the missing points in this article are the absence of performance indices for each specific application but also the artibution of usable technologies is attributed only to the category and not to the specific applications.


%%{\bf NS: pas sûr de garder, materiel de formation, pas la source primaire) Other research accentuates this research by also integrating KPIs, such as the ITS United States Department of Transportation \cite{noauthor_its_nodate} \cite{noauthor_its_nodate-1}.} %But in their approaches, there is only the consideration of a small part of specific applications \cite{lee_latency_2017}, such as security applications, but in our case we wish to access our studies in the case of a more exhaustive panel of 'transport applications.

%Still others assimilate to each specific application, a usable technology. But again, information on KPIs is lacking and the number of specific applications is limited.  \cite{boban_use_2017}

%There are obviously other classifications in the literature, whether in telecommunications or transport, or both. But the flaws are similar to those mentioned above, there is no completely exhaustive list for each application, at least in the affiliation of the technologies used. \cite{raza_social_2018} \cite{karagiannis_vehicular_2011}

Another issue is related to the fast pace of innovation and change in this field, with some technologies becoming quickly obsolete, new ones being deployed and applications becoming common place, but through other technologies and devices than envisioned initially. A good example are vehicle-based instant messaging and ``insurance and financial services''\cite{etsi_etsi_tr_102_638_intelligent_2009} applications: the former started on computers and moved to smartphones, while the latter generally started as dedicated global navigation satellite service (GNSS) devices plugged to the on-board diagnostic ports in vehicles, then moved to smartphones as well. Their ubiquitous nature and the ease to develop geo-located user applications make smartphones a strong challenger for the platform of choice of many \acrshort{CT} applications. That is why even relatively recent taxonomies may already be partially obsolete. 

% In addition, with technological developments and the new challenges of 5G, some applications can sometimes be obsolete or even useless today.

This review of existing taxonomies and classifications of \acrshort{CT} applications shows a clear gap in the literature. On the transportation side, the categories are often partial, too broad or exclusive, where an application can belong to only one category. On the telecommunication side, the \acrshort{KPI}s and \acrshort{KPI} requirements for each application are often missing or partial. The two fields, transportation and telecommunication, are not completely mapped. To address the current gaps, the existing taxonomies were therefore integrated and combined into the taxonomy proposed in this paper, meant to be as complete as possible. 

% Furthermore, not all of these taxonomies have the considerable and necessary information for intelligent transport issues. Namely for each application to have both exhaustive information in transport but also in telecommunications.

% Some applications are now obsolete because they no longer respond to expectations or no longer pass through connected cars. Indeed, we are currently witnessing the transfer of certain transport applications to smartphone applications.

% We have therefore taken in our taxonomy to take into account only the applications that make sense today and only use vehicular communications. % NS: no

% What is the impact on mobility? The applications are divided into several parts, with mobility applications, vehicle connectivity applications, intelligent parking applications and messaging network service applications. We can notice in these examples of applications that we thought to be treated by cars are now used by smartphone

% But this is the case for applications that have an indirect impact on the mobility transferred to smartphones. This is mainly the case for insurance applications but also in some cases for energy consumption applications. 

% However, we were able to derive a lot of useful information from these classifications and other articles that do not consider the classification:

% \begin{itemize}
% \item In transport with the directory of specific applications in the relevant and recurring category in this area, such as safety or efficiency.
% \item At the heart of telecommunications which have made it possible to derive the various information relating to the exchange of information and communications for connected vehicles
% \end{itemize}

% In this study, we will therefore attempt to build as precise a list as possible of transport applications specific to connected vehicles. Consideration of transport classification information and its attribute when collecting telecommunication information about deployable applications for proper implementation.

% There is a fairly significant deficit for firm and the entire scientific field in this type of study. Indeed, this type of research is a useful classification both for questions relating to the Internet of Things or smart cities, but also for studies of automated vehicles.
